\documentclass[11pt,a4paper]{beamer}
\usepackage[utf8x]{inputenc}
\usepackage{amsmath}
\usepackage{amsfonts}
\usepackage{amssymb}
\usepackage{listings} %For syntax highlight
\usepackage[L7x]{fontenc}
\usepackage[lithuanian]{babel}
\usepackage{graphics}
\usepackage{hyperref}

\usetheme{Antibes} 
\usecolortheme[RGB={155,192,12}]{structure} 
%\usecolortheme{albatros}

\author{Povilas Balzaravičius}
\title{Prezentacijų kūrimas su LaTeX Beamer}
\subtitle{Ubuntu 10.10 release party}

\begin{document}

\begin{frame}
	\titlepage
\end{frame}


\begin{frame}{Kas aš?}
    \begin{itemize}
        \item Povilas Balzaravičius
        \item Internete dar žinomas kaip Pawka
        \item Web programuotojas
        \item pavvka@gmail.com (+Jabber)
        \item \href{http://pawka.linija.net/}{http://pawka.linija.net/}
    \end{itemize}  
\end{frame}


\begin{frame}{Turinys}
	\tableofcontents
\end{frame}


\section{Įžanga}
\begin{frame}{Apie ką mes čia?}
    \begin{itemize}
        \item LaTeX - žymėjimo kalba ir sistema, skirta dokumentų rengimui.
        \item Beamer klasė - skirta prezentacijų rengimui LaTeX kalba.
    \end{itemize}
\end{frame}


\begin{frame}{Kodėl LaTeX Beamer?}
    \begin{itemize}
        \item Nereikia rūpintis atvaizdavimu.
        \item Tekstinis \textit{plain-text} turinys.
        \item Suderinama su versijų kontrolės sistemomis.
        \item Galima naudoti mėgstamą teksto redaktorių.
        \item Programinio kodo žymėjimas \textit{(highlight)}.
        \item Automatinė numeracija, turinys, ...
        \item Nėra atvaizdavimo problemų (pdf).
    \end{itemize}
\end{frame}


\section{Programinė įranga}
\begin{frame}{Kaip pradėti naudotis?}
    \begin{center}
        \textbf{sudo apt-get install texlive littex}
        \vskip15pt
        \pause
        ...ir bet kuris teksto redaktorius, pvz.\\
        \textbf{sudo apt-get install texmaker}
        \vskip15pt
        \pause
        \alert{Viso apie 700 MB. Mykolas prašė nesisiųsti :-)}
    \end{center}
\end{frame}

\section{Prezentacijos kūrimas}
\subsection{Temos}
\begin{frame}[fragile]{Temų naudojimas}
    \begin{itemize}
        \item Beamer turi iš anksto paruoštas temas, kurios leidžia greitai pakeisti išvaizdą.
        \item Trys tipai: išdėstymas, spalvos ir šriftai.
        \item Naudojamos komantos: \verb|\usetheme{Tema}| ir \verb|\usecolortheme{Tema}|.
        \item Žinoma jas galima redaguoti, bet tai išsiaiškinsit patys :-)
        \item Temų sąrašas: \href{http://www.hartwork.org/beamer-theme-matrix/}{http://www.hartwork.org/beamer-theme-matrix/}
    \end{itemize}
\end{frame}


\subsection{Skaidrės}
\begin{frame}[fragile]
    \frametitle{Skaidrės kūrimas}
    Prezentacija susideda iš skaidrių :-) 
    \vskip10pt
    \begin{block}{Skaidrės kūrimas}
        \begin{verbatim}
        \begin{frame}
            \frametitle{Antraštė}
            Tekstas arba LaTeX kodas.
        \end{frame}
        \end{verbatim}
    \end{block}
    Ir gausim kažką panašaus į...
\end{frame}
\begin{frame}
    \frametitle{Antraštė}
    Tekstas arba LaTeX kodas.
\end{frame}

\subsection{Skyriai ir turinys}
\begin{frame}[fragile]{Skyrių naudojimas}
    \begin{itemize}
        \item Tvarkingas dokumentas susideda iš skyrių.
        \item Iš jų generuojamas turinys.
        \item Beamer palaiko trijų lygių skyrius:
        \begin{itemize}
            \item \verb|\section{Pavadinimas}|
            \item \verb|\subsection{sub-Pavadinimas}|
            \item \verb|\subsubsection{sub-sub-Pavadinimas}|
        \end{itemize}
    \end{itemize}
\end{frame}
\begin{frame}[fragile]{Turinio generavimas}
    Tvarkingai surašius skyrius, galima sugeneruoti turinį.
    \begin{block}{Kodas}
\verb|\begin{frame}|\\
\verb|\tableofcontents|\\
\verb|\end{frame}|
    \end{block}
\end{frame}
\begin{frame}
    \tableofcontents
\end{frame}

\subsection{Sąrašai}
\begin{frame}
    \frametitle{Sąrašai}
    Kurdami prezentacijas dažnai naudojame sąrašus. Beamer turi tris sąrašų tipus:
    \begin{description}
        \item[itemize] Paprastas sąrašas
        \item[enumerate] Sunumeruotas sąrašas
        \item[description] Elementų sąrašas, su paaiškinimais. Šis sąrašas yra description tipo.
    \end{description}
\end{frame}
\begin{frame}[fragile]
    \frametitle{Paprastas sąrašas}
    Sąrašų struktūra aprašoma panašiai. Kiekvienas sąrašo elementas pradedamas
    \verb|\item| komanda. Paprastas sąrašas aprašomas taip:
    \begin{columns}[t]
        \begin{column}{5cm}
            \begin{block}{Kodas}
                \begin{lstlisting}
\begin{itemize}
\item Ubuntu
\item Kubuntu
\item Xubuntu 
\end{itemize}
                \end{lstlisting}
            \end{block}
        \end{column}
        \begin{column}{5cm}
            \begin{block}{Rezultatas}
                \begin{itemize}
                    \item Ubuntu
                    \item Kubuntu
                    \item Xubuntu 
                \end{itemize}
            \end{block}
        \end{column}
    \end{columns}
\end{frame}
\begin{frame}[fragile]
    \frametitle{Sunumeruotas sąrašas}
    Sunumeruotą sąrašą gausime raktinį žodį itemize pakeitę į enumerate.
    \begin{columns}[t]
        \begin{column}{5cm}
            \begin{block}{Kodas}
                \begin{lstlisting}
\begin{enumerate}
\item Ubuntu
\item Kubuntu
\item Xubuntu 
\end{enumerate}
                \end{lstlisting}
            \end{block}
        \end{column}
        \begin{column}{5cm}
            \begin{block}{Rezultatas}
                \begin{enumerate}
                    \item Ubuntu
                    \item Kubuntu
                    \item Xubuntu 
                \end{enumerate}
            \end{block}
        \end{column}
    \end{columns}
\end{frame}
\begin{frame}[fragile]
    \frametitle{Elementų sąrašas}
    Description tipas neženkliai skiriasi nuo anksčiau minėtų. Prie kiekvieno
    sąrašo elemento yra nurodomas pavadinimas ir jo paaiškinimas.
    \begin{columns}[t]
        \begin{column}{5cm}
            \begin{block}{Kodas}
                \begin{lstlisting}
\begin{description}
\item[Gnome] Ubuntu
\item[KDE] Kubuntu
\item[XFCE] Xubuntu 
\end{description}
                \end{lstlisting}
            \end{block}
        \end{column}
        \begin{column}{5cm}
            \begin{block}{Rezultatas}
                \begin{description}
                \item[Gnome] Ubuntu
                \item[KDE] Kubuntu
                \item[XFCE] Xubuntu 
                \end{description}
            \end{block}
        \end{column}
    \end{columns}
\end{frame}

\subsection{Tekstas ir lygiavimas}
\begin{frame}[fragile]
    \frametitle{Lygiavimas}
    Galimi trys lygiavimo variantai:
    \begin{description}
    	\item[flushleft] Kairė
    	\item[flushright] Dešinė
    	\item[center] Centras
    \end{description}  
    \begin{columns}[t]
        \begin{column}{5cm}
    \begin{block}{Teksto centravimas}
        \begin{lstlisting}
			\begin{center}
			Tekstas centre.
			\end{center}
        \end{lstlisting}
    \end{block}
        \end{column}
        \begin{column}{5cm}
            \begin{block}{Rezultatas}
                \begin{center}
                Tekstas centre.
                \end{center}
            \end{block}
        \end{column}
    \end{columns}    
\end{frame}

\begin{frame}[fragile]
	\frametitle{Atitraukimai}
	\begin{itemize}
		\item Teksto ir kitų objektų atitraukimui naudojamos \verb|\vskip| ir \verb|\hskip| komandos (\textit{vertical} ir \textit{horisontal}).
		\item Komandos rašomos pagal šabloną \verb|\vskip<kiekis><vienetai>|.
		\item Kiekis - skaičius, kuris nurodo kokiu atstumu atitraukti objektą. Gali būti ir neigiamas.
		\item Vienetai: pt, cm (kiti?).  Nurodo kokiais vienetais matuoti atitraukimą.
		\item Pavyzdžiai: \verb|\vskip10pt| \verb|\hskip-2cm|
	\end{itemize}
\end{frame}


\begin{frame}[fragile]{Atitraukimo pavyzdžiai}
\hskip1cmPrieš šį tekstą parašyta \verb|\hskip1cm|.\\
\pause
\hskip2cmPrieš šį tekstą parašyta \verb|\hskip2cm|.\\
\pause
\hskip3cmPrieš šį tekstą parašyta \verb|\hskip3cm|.\\
\pause
\hskip4cmPrieš šį tekstą parašyta \verb|\hskip4cm|.\\
\pause
\vskip30ptPrieš šį tekstą parašyta \verb|\vskip30pt|.
\pause
\vskip-65pt
\color{blue}{Prieš šį tekstą parašyta \verb|\vskip-65pt|.}
\end{frame}


\begin{frame}[fragile]
	\frametitle{Teksto formatavimas}
	Pagrindinės teksto formatavimo komandos:
	
	\begin{block}{Teksto komandos}
		\begin{columns}[t]
			\begin{column}{5cm}
				\verb|\textbf{Labas}|\\
				\verb|\textit{Labas}|\\
				\verb|\textsc{Labas}|\\
				\verb|\textsl{Labas}|\\
				\verb|\texttt{Labas}|\\
				\verb|\alert{Labas}|\\
				\verb|\color{orange}{Labas}|\\
				\verb|\structure{Labas}|
			\end{column}
			\begin{column}{5cm}
				\textbf{Labas}\\
				\textit{Labas}\\
				\textsc{Labas}\\
				\textsl{Labas}\\
				\texttt{Labas}\\
				\alert{Labas}\\
				\color{orange}{Labas}\\
				\structure{Labas}
			\end{column}
		\end{columns}
	\end{block}
\end{frame}


\subsection{Kodo atvaizdavimas}
\begin{frame}[fragile]{Listings paketas kodo atvaizdavimui}
    \begin{itemize}
        \item Dokumento pradžioje nurodom \verb|\usepackage{listings}|.
        \item Leidžia atvaizduoti programavimo kalbų kodą.
        \item Kodą galima įtraukti iš failo.
        \item Palaiko daaaaugelio kalbų sintaksę.
        \item Turi dar daugiau galimybių (eilučių numeravimas, atitraukimai, ...)
        \item \href{http://en.wikibooks.org/wiki/LaTeX/Packages/Listings
}{http://en.wikibooks.org/wiki/LaTeX/Packages/Listings}
    \end{itemize}
\end{frame}

\begin{frame}[fragile]{Kodo vaizdavimas dokumente}
	\begin{block}{Kodas}
\verb|\begin{lstlisting}[language=python]|\\
\verb|def returnFalse():|\\
\verb|    return False|\\
\verb|\end{lstlisting}|
	\end{block}
	\begin{block}{rezultatas}
    \begin{lstlisting}[language=python]
def returnFalse():
    return False
    \end{lstlisting}
	\end{block}
\end{frame}
\begin{frame}[fragile]{Kodo įtraukimas iš išorinių failų}
    Kodą galima įtraukti iš išorinių failų. Patogu, nes kodas turi būti savo vietoje :-)
    \begin{lstlisting}
\lstset{language=python}
    \lstinputlisting[firstline=2,lastline=7]...
       ...{code/009.py}
    \end{lstlisting}
    \pause
    \lstset{language=python}
        \lstinputlisting[firstline=2,lastline=7]{code/009.py}
\end{frame}

\subsection{Atvaizdavimo eiliškumas}
\begin{frame}{Atvaizdavimo eiliškumas, efektai arba overlays}
	\begin{itemize}
		\item WYSIWYG programos turi priemones, leidžiančias skaidrės objektus atvaizduoti tam tikra tvarka. Dažniausiai jos paremtos grafiniais efektais.
		\item Beamer turi priemones, kurios taip pat leidžia nurodyti tvarką, kuria bus vaizduojami objektai.		
		\item Elementai palaipsniui atidengiami naujuose PDF failo puslapiuose.
		\item Angliškai \textit{overlays}, tačiau aš vadinu efektais :-)
	\end{itemize}
\end{frame}

\begin{frame}{Dokumento struktūra}
	\begin{itemize}
		\item Prezentaciją sudaro skaidrės.
		\item Tą pačią skaidrę gali sudaryti keli pdf dokumento puslapiai (vadinkime juos žingsniais).
		\item Taip galima vienoje skaidrėje elementus atvaizduoti ne visus iš karto.
	\end{itemize}
\end{frame}


\begin{frame}[fragile]{Pauzė}
	Jei objektai išdėlioti paeiliui, galima naudoti komandą \verb|\pause|. Tekstas, esantis po šios komanods, bus atvaizduotas kitame dokumento puslapyje.
	\begin{block}{Kodas}
		Eins \begin{color}{red}\textbackslash pause\end{color} zwei \begin{color}{red}\textbackslash pause\end{color} drei
	\end{block}
	\pause
	\begin{block}{rezultatas}
		Eins \pause zwei \pause drei
	\end{block}
\end{frame}
\begin{frame}{Atvaizdavimo specifikacijos}
	\begin{itemize}
		\item Ką daryti jei norime elementus atvaizduoti ne paeiliui?
		\item Galime naudoti atvaizdavimo specifikacijas \textit{(overlay specifications)}.
		\item Jų dėka galima nurodyti kuriuo metu kuriuos objektus atvaizduoti.
	\end{itemize}
\end{frame}
\begin{frame}[fragile]{Atvaizdavimo specifikacijos - formatas}
	\begin{itemize}
		\item Specifikacija aprašoma tarp ženklų \verb|<|, \verb|>|.
		\item Tarp ženklų nurodoma kuriuo metu atvaizduoti objektą.
		\item <1> - atvaizdavimas pirmame žingsnyje, <2-> - nuo antro žingsnio iki skaidrės pabaigos, <-3>, <2-4>, <1,3,4>...
	\end{itemize}
\end{frame}
\begin{frame}[fragile]{Atvaizdavimo specifikacijos - pavyzdys}
    \begin{columns}[t]
        \begin{column}{5cm}
            \begin{block}{Kodas}
\begin{verbatim}
\begin{itemize}
    \item<1-> Ubuntu
    \item<3,4> Kubuntu
    \item<-3> Xubuntu 
    \item<2-3> Edubntu
\end{itemize}
\end{verbatim}
            \end{block}
        \end{column}
        \begin{column}{5cm}
            \begin{block}{Rezultatas}
                \begin{itemize}
                    \item<1-> Ubuntu
                    \item<3,4> Kubuntu
                    \item<-3> Xubuntu 
                    \item<2-3> Edubntu
                \end{itemize}
            \end{block}
        \end{column}
    \end{columns}
\end{frame}
\begin{frame}[fragile]{Atvaizdavimo specifikacijos - pavyzdys}
    Galima naudoti ne tik sąrašams.
    \begin{columns}[t]
        \begin{column}{5cm}
            \begin{block}{Kodas}
\begin{verbatim}
\alert<1->{Ubuntu}\\
\alert<3,4>{Kubuntu}\\
\alert<-3>{Xubuntu}\\
\alert<2-3>{Edubntu}\\
\end{verbatim}
            \end{block}
        \end{column}
        \begin{column}{5cm}
            \begin{block}{Rezultatas}
                \alert<1->{Ubuntu}\\
                \alert<3,4>{Kubuntu}\\
                \alert<-3>{Xubuntu}\\
                \alert<2-3>{Edubntu}\\
            \end{block}
        \end{column}
    \end{columns}
\end{frame}


\subsection{Grafika}
\begin{frame}[fragile]{Grafinių elementų įkėlimas}
    \begin{itemize}
        \item Naudosim graphics paketą \verb|\usepackage{graphics}|.
        \item Palaiko jpg, gif, png formatus.
        \item \verb|\includegraphics[width=4cm]{img/berlin.png}|
    \end{itemize}
    \begin{center}
    \includegraphics[width=4cm]{img/berlin.png} 
    \end{center}
\end{frame}

\subsection{Lietuvybė}
\begin{frame}[fragile]{Lietuviškų simbolių naudojimas}
    Norėdami naudoti lietuviškus simbolius, turime dokumento pradžioje įtraukti šias kodo eilutes:
    \begin{verbatim}
    \usepackage[L7x]{fontenc}
    \usepackage[lithuanian]{babel} 
    \end{verbatim}
\end{frame}

\section{Pabaiga}
\begin{frame}{Resursai}
    \begin{itemize}
        \item Beamer - \href{http://bitbucket.org/rivanvx/beamer/}{http://bitbucket.org/rivanvx/beamer/} 
        \item Daug info - \href{http://en.wikibooks.org/wiki/LaTeX}{http://en.wikibooks.org/wiki/LaTeX}
        \item Ši prezentacija - \href{http://bitbucket.org/pawka/keynotes/}{http://bitbucket.org/pawka/keynotes/}
        \item TeX, LaTeX and Friends (Q\&A) - \href{http://tex.stackexchange.com/}{http://tex.stackexchange.com/} 
    \end{itemize}
\end{frame}


\begin{frame}
	\begin{center}
	Ačiū
	\end{center}
\end{frame}


\end{document}
