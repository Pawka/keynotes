\documentclass[12pt,a4paper]{beamer}
\usepackage{graphicx}
\usepackage{hyperref}
\usepackage{lmodern}
\usepackage{listings}
\usepackage[utf8x]{inputenc}
\usepackage[L7x]{fontenc}
\usepackage[lithuanian]{babel}
\usepackage{minted}

% Theme
\usetheme{Antibes} 
\usecolortheme[RGB={155,192,12}]{structure} 

\definecolor{codebg}{rgb}{0.10, 0.11, 0.08}
\definecolor{bashbg}{rgb}{0.95,0.95,0.95}
\definecolor{xmlbg}{rgb}{0.20, 0.22, 0.18}

% Code highlight setup
\usemintedstyle{monokai}
\newminted{php}{
    linenos,
    bgcolor=codebg,
    firstline=2,
    fontsize=\scriptsize,
    gobble=4
}
\newminted{bash}{
    bgcolor=bashbg,
    fontsize=\scriptsize,
    gobble=4
}
\newminted{xml}{
    linenos,
    bgcolor=codebg,
    fontsize=\scriptsize,
    gobble=4
}
\newminted{yaml}{
    linenos,
    bgcolor=codebg,
    fontsize=\scriptsize,
    gobble=4
}

\author{Povilas Balzaravičius}
\title{Dependency Injection Containers}
\subtitle{VilniusPHP Susitikimas \#3}

\begin{document}
\section{Įžanga}
\begin{frame}
	\titlepage
\end{frame}

\begin{frame}{Kas aš toks?}
    \begin{itemize}
        \item Povilas Balzaravičius
        \item \href{https://twitter.com/pawka}{@Pawka}
        \item \href{https://github.com/pawka}{github.com/pawka}
        \item \href{https://linkedin.com/in/pawka}{linkedin.com/in/pawka}
        \item \href{http://pawka.linija.net}{pawka.linija.net}
    \end{itemize}
    \begin{center}
        \includegraphics[scale=0.4]{img/estina.png}
        \hskip1.5cm
        \includegraphics[scale=0.4]{img/zce.png}
        \hskip1.5cm
        \includegraphics[scale=0.75]{img/ktu.png}
    \end{center}
\end{frame}

\section{Dependency Injection}
\subsection{Kas yra Dependency Injection?}
\begin{frame}
	\begin{center}
        {\Huge Kas yra Dependency Injection?}
	\end{center}
\end{frame}


\begin{frame}[fragile]{Kuo blogas šis kodas?}
\begin{phpcode}
    <?php
    class FeedGenerator {

        protected $feed;

        public function __construct() {
            // Create feed every 6 hours.
            $this->feed = new SomeFeed('0 */6 * * *');
        }
    }
    //..
    $generator = new FeedGenerator;
\end{phpcode}
\pause
\begin{enumerate}
    \item Sudėtinga parašyti testą.
    \item Gali tekti keisti kodą modifikuojant Feed klasę.
    \item Teks modifikuoti FeedGenerator klasę, norint pakeisti SomeFeed vykdymo intervalą.
\end{enumerate}
\end{frame}

\begin{frame}[fragile]{Ar dabar geriau?}
\begin{phpcode}
    <?php
    class FeedGenerator {

        protected $feed;

        public function __construct($interval) {
            $this->feed = new SomeFeed($interval);
        }
    }
    //..
    $generator = new FeedGenerator('0 */6 * * *');
\end{phpcode}
\pause
\begin{enumerate}
    \item Sudėtinga parašyti testą.
    \item Gali tekti keisti kodą modifikuojant Feed klasę.
    \item SomeFeed klasė vis dar priklausoma nuo FeedGenerator.
\end{enumerate}
\end{frame}


\begin{frame}[fragile]{Ką daryti?}
    \begin{center}
        {\Huge Iškelti vidinių objektų kūrimą už klasės ribų!}
    \end{center}
\end{frame}


\begin{frame}[fragile]{Kodėl šis kodas geresnis?}
\begin{phpcode}
    <?php
    namespace Feed;

    class FeedGenerator {

        protected $feed;

        public function __construct(FeedInterface $feed) {
            $this->feed = $feed;
        }
    }
    //..
    $feed = new SomeFeed('0 */6 * * *');
    $generator = new FeedGenerator($feed);
\end{phpcode}
\begin{enumerate}
    \item Patogu rašyti testus (vietoj Feed galima paduoti mock objektą).
    \item \$feed objektas nepriklauso nuo FeedGenerator klasės.
    \item Modifikuojant FeedInterface klases, nereikės keisti FeedGenerator kodo.
\end{enumerate}
\end{frame}

\begin{frame}[fragile]{Sveiki, aš DI}
    \begin{center}
        {\Huge Štai Jums Dependency Injection!}
    \end{center}
\end{frame}

\begin{frame}[fragile]{Objektų kūrimas}
    DI naudojimui konteineris nereikalingas.
    \vskip0.5cm
\begin{phpcode}
    <?php
    $config = new \Doctrine\DBAL\Configuration();
    $connectionParams = array(
        'dbname' => 'mydb',
        'user' => 'user',
        'password' => 'secret',
        'host' => 'localhost',
        'driver' => 'pdo_mysql',
    );
    $conn = \Doctrine\DBAL\DriverManager::getConnection(
        $connectionParams, $config);
    $productService = new Product($conn);
    $interval = "0 */6 * * *";
    $feed = new \Feed\SomeFeed($interval);
    $feed->setProductService($productService);
    $feedGenerator = new \Feed\FeedGenerator($feed);
    //Maybe add objects to some registry.
    //..

\end{phpcode}
\end{frame}

\begin{frame}{Kas yra DI?}

    \begin{itemize}
        \item Projektavimo šablonas \emph{(design pattern)}.
        \item Aktualus tik objektiniame programavime.
        \item Atskiria komponentus \emph{(decoupling)}.
        \item Standartizuoja ir suteikia galimybę centralizuoti objektų kūrimą Jūsų sistemoje.
    \end{itemize}
\end{frame}

\begin{frame}[fragile]{DI būdai}
\begin{phpcode}
    <?php
    $feed = new Feed("0 */6 * * *");

    //Konstruktorius
    $generator = new FeedGenerator($feed);

    //Metodas (setter)
    $generator = new FeedGenerator;
    $generator->setFeed($feed);

    //Atributas
    $generator = new FeedGenerator;
    $generator->feed = $feed;
\end{phpcode}
\end{frame}

\section{DIC bibliotekos}
\begin{frame}
	\begin{center}
        {\footnotesize Kam kurti savo jei yra..}\\*
        \vskip0.5cm
        {\Huge DIC bibliotekos}
	\end{center}
\end{frame}

\begin{frame}{Reikalavimai DI konteineriui}
    \begin{itemize}
        \item Turi veikti greitai.
        \item Privalo dirbti su bet kokiu PHP objektu.
        \item Objektai neturi žinoti apie DI konteinerį.
        \item Nekurti objekto jei jis jau sukurtas.\footnote{Kai to reikia.}
    \end{itemize}
\end{frame}

\subsection{Pimple}
\begin{frame}
	\begin{center}
        {\Huge Pimple}
	\end{center}
\end{frame}

\begin{frame}{Pimple}
    A simple Dependency Injection Container for PHP 5.3.
    \begin{itemize}
        \item Autorius: Fabien Potencier
        \item \url{pimple.sensiolabs.org}
        \item Paprastas.
        \item Greitas.
        \item Integravimas - kelios eilutės.
    \end{itemize}
\end{frame}

\begin{frame}[fragile]{Darbas su objektais}
\begin{phpcode}
    <?php
    $container = new Pimple();
    // ..

    $container['feed.somefeed.interval'] = '0 */6 * * *';
    $container['feed.somefeed.class'] = '\Feed\SomeFeed';

    $container['feed.somefeed'] = function($c) {
        return new $c['feed.somefeed.class'](
            $c['feed.somefeed.interval']
        );
    };

    //Lazy loading
    $container['feed.generator.class'] = '\Feed\FeedGenerator';
    $container['feed.generator'] = $container->share(function($c) {
        return new $c['feed.generator.class']($c['feed.somefeed']);
    });

    // ..
    $manager = $container['feed.generator'];
\end{phpcode}
\end{frame}

\begin{frame}[fragile]{Išskaidymas į modulius}
\begin{phpcode}
    <?php
    class FeedContainer extends Pimple {

        public function __construct() {
            $this['feed.somefeed.interval'] = '0 */6 * * *';
            $this['feed.somefeed.class'] = '\Feed\SomeFeed';

            $this['feed.somefeed'] = function($c) {
                return new $c['feed.somefeed.class'](
                    $c['feed.somefeed.interval']
                );
            };

            $this['feed.generator.class'] = '\Feed\FeedGenerator';
            $this['feed.generator'] = $this->share(function($c) {
                return new $c['feed.generator.class']($c['feed.somefeed']);
            });
        }
    }
\end{phpcode}
\end{frame}

\begin{frame}[fragile]{Išskaidymas į modulius}
\begin{phpcode}
    <?php
    class GlobalContainer extends Pimple {

        public function __construct() {
            $this['feeds'] = $this->share(function($c) {
                return new FeedContainer;
            };
        }
    }

    //..
    $container = new GlobalContainer;
    //Change value
    $container['feeds']['feed.somefeed.interval'] = '0 */12 * * *';
    $generator = $container['feeds']['feed.generator'];
\end{phpcode}
\end{frame}


\begin{frame}[fragile]{Sparta}
    Konteinerio kūrimas
    \vskip0.5cm
\begin{bashcode}
    ./run-tests.sh 
    Pimple
    265K
    579K
    0.00086s
    DependencyInjection
    267K
    931K
    0.006124s
    DependencyInjection (dumped)
    688K
    738K
    0.000495s
\end{bashcode}
    \vskip0.25cm
    \scriptsize{Šaltinis: \url{https://gist.github.com/igorw/3833123}}
\end{frame}


\subsection{Symfony Dependency Injection Component}
\begin{frame}
	\begin{center}
        {\Huge Symfony Dependency Injection Component}
	\end{center}
\end{frame}

\begin{frame}{Symfony DIC}
    \begin{itemize}
        \item Symfony komponentas.
        \item \url{http://symfony.com/components}
        \item Daug galimybių.
        \item Greitas (teisingai naudojant).
        \item Pagal nutylėjimą naudojamas Symfony2 karkase.
    \end{itemize}
\end{frame}

\begin{frame}{Palaikomi formatai}
    Prikalusomybes galima aprašyti šiais formatais:
    \begin{itemize}
        \item PHP
        \item XML
        \item YAML
        \item INI (Palaiko tik parametrų aprašymus).
    \end{itemize}
\end{frame}

\begin{frame}[fragile]{Naudojimas: PHP}
\begin{phpcode}
    <?php
    use Symfony\Component\DependencyInjection\ContainerBuilder;
    use Symfony\Component\DependencyInjection\Reference;

    $container = new ContainerBuilder();

    $container->setParameter('feed.somefeed.interval', '0 */6 * * *');
    $container
        ->register('feed.somefeed', 'SomeFeed')
        ->addArgument('%feed.somefeed.interval%');

    $container
        ->register('feed.generator', 'FeedGenerator')
        ->addArgument(new Reference('feed.somefeed'));
\end{phpcode}
\end{frame}

\begin{frame}[fragile]{Naudojimas: XML}
\begin{xmlcode}
    <parameters>
        <parameter key="feed.somefeed.interval">0 */6 * * *</parameter>
    </parameters>

    <services>
        <service id="feed.somefeed" class="SomeFeed">
            <argument>%feed.somefeed.interval%</argument>
        </service>

        <service id="feed.generator" class="FeedGenerator">
            <argument type="service" id="feed.somefeed" />
        </service>
    </services>
\end{xmlcode}
\end{frame}

\begin{frame}[fragile]{Naudojimas: YAML}
\begin{yamlcode}
    parameters:
        # ...
        feed.somefeed.interval: 0 */6 * * *

    services:
        feed.somefeed:
            class:     SomeFeed 
            arguments: [%feed.somefeed.interval%]
        feed.generator:
            class:     FeedGenerator
            arguments: [@feed.somefeed]
\end{yamlcode}
\end{frame}


\begin{frame}[fragile]{DI perdengimas}
    \begin{itemize}
        \item Palaiko YAML ir XML formatai.
        \item Leidžia perdengti anksčiau aprašytas DI priklausomybes (pvz. dirbant su Symfony2).
        \item Patogu naudoti skirtingas konfigūracijas: test, dev, live.
    \end{itemize}
    \vskip0.5cm
\begin{phpcode}
    <?php
    use Symfony\Component\DependencyInjection\ContainerBuilder;
    use Symfony\Component\Config\FileLocator;
    use Symfony\Component\DependencyInjection\Loader\XmlFileLoader;

    $container = new ContainerBuilder();
    $loader = new XmlFileLoader($container, new FileLocator(__DIR__));
    $loader->load('services.xml');
\end{phpcode}
\end{frame}

\begin{frame}[fragile]{Kompiliavimas}
    \begin{itemize}
        \item Konteineris \textbf{gali} būti kompiliuojamas.
\begin{phpcode}
    <?php
    $container->compile();
\end{phpcode}
        \item Aptinkamos klaidos:
            \begin{itemize}
                \item Neegzistuojančios priklausomybės.
                \item Ciklinės priklausomybės.
            \end{itemize}
        \item Atliekama optimizacija.
        \item Sukompiliuotas rezultatas gali būti saugomas į diską ir naudojamas kaip \emph{cache'as}.\footnote{Pamenat spartos palyginimo rezultatus?}
    \end{itemize}
    \vskip0.5cm
    P.S. Symfony2 tuo rūpinasi pagal nutylėjimą :-)
\end{frame}

\begin{frame}[fragile]{Sukompiliuoto rezultato saugojimas į diską}
    \begin{itemize}
        \item Naudojamos Dumper klasės.
        \item Galima konvertuoti iš vieno formato į kitą.
        \item Palaikomi formatai:
            \begin{itemize}
                \item PHP
                \item XML
                \item YAML
                \item GraphWiz (\url{www.graphwiz.org})
            \end{itemize}
    \end{itemize}
\end{frame}

\begin{frame}[fragile]{Sukompiliuoto rezultato saugojimas į diską}
\begin{phpcode}
    <?php
    use Symfony\Component\DependencyInjection\ContainerBuilder;
    use Symfony\Component\DependencyInjection\Dumper\PhpDumper;

    $file = __DIR__ .'/cache/container.php';

    if (file_exists($file)) {
        require_once $file;
        $container = new ProjectServiceContainer();
    } else {
        $container = new ContainerBuilder();
        // ...
        $container->compile();

        $dumper = new PhpDumper($container);
        file_put_contents($file, $dumper->dump());
    }
\end{phpcode}
\vskip0.5cm
ProjectServiceContainer - pavadinimas pagal nutylėjimą.
\end{frame}

\begin{frame}[fragile]{Tagged servisai}
    \begin{itemize}
        \item \emph{Tag} - atributas, žymintis panašius objektus.
        \item Leidžia vykdyti analogiškus veiksmus visiems pažymėtiems objektams.
    \end{itemize}
\end{frame}

\begin{frame}[fragile]{Tagged servisai}
    Modifikuojam FeedGenerator klasę, pridėdami kelių Feed objektų palaikymą:
    \vskip0.5cm
\begin{phpcode}
    <?php
    class FeedGenerator {

        protected $feeds = array();

        public function __construct() {
        }

        public function addFeed(FeedInterface $feed) {
            $this->feeds[] = $feed;
        }
    }
\end{phpcode}
\end{frame}


\begin{frame}[fragile]{Tagged servisai}
    Turim aprašytas kelias skirtingas Feed klases.
    \vskip0.5cm
\begin{xmlcode}
    <parameters>
        <parameter key="feed.somefeed.interval">0 */6 * * *</parameter>
        <parameter key="feed.otherfeed.interval">0 */12 * * *</parameter>
    </parameters>

    <services>
        <service id="feed.somefeed" class="SomeFeed">
            <argument>%feed.somefeed.interval%</argument>
            <tag name="feed" />
        </service>
        <service id="feed.otherfeed" class="OtherFeed">
            <argument>%feed.otherfeed.interval%</argument>
            <tag name="feed" />
        </service>

        <service id="feed.generator" class="FeedGenerator" />
    </services>
\end{xmlcode}
\end{frame}

\begin{frame}[fragile]{Tagged servisai}
\begin{phpcode}
    <?php
    use Symfony\Component\DependencyInjection\ContainerBuilder;
    use Symfony\Component\DependencyInjection\Compiler\CompilerPassInterface;
    use Symfony\Component\DependencyInjection\Reference;

    class FeedCompilerPass implements CompilerPassInterface
    {
        public function process(ContainerBuilder $container)
        {
            if (!$container->hasDefinition('feed.generator')) {
                return;
            }

            $definition = $container->getDefinition('feed.generator');
            $taggedServices = $container->findTaggedServiceIds('feed');
            foreach ($taggedServices as $id => $attributes) {
                $definition->addMethodCall('addFeed',
                    array(new Reference($id)));
            }
        }
    }
\end{phpcode}
\end{frame}

\section{Pabaiga}

\subsection{Išvados}
\begin{frame}{Išvados}
    \begin{itemize}
        \item Norint pradėti naudoti DI, neprivaloma naudoti tam skirtos bibliotekos.
        \item DI verčia kodą rašyti teisingai.
        \item Paprasčiau atlikti pakeitimus.
        \item Paprasčiau (įmanoma?) rašyti testus.
        \item Ar verta DI naudoti egzistuojančiuose projektuose? TAIP (žr. aukščiau esančius punktus).
    \end{itemize}
\end{frame}

\subsection{Resursai}
\begin{frame}{Resursai}
    \begin{itemize}
        \item Symfony Dependency Injection component docs: \url{http://symfony.com/doc/current/components/dependency_injection/}
        \item Fabien Potencier blog: What is dependency injection? \url{http://fabien.potencier.org/article/11/what-is-dependency-injection}
        \item Inversion of Control Containers and the Dependency Injection pattern: \url{http://www.martinfowler.com/articles/injection.html}
        \item Zend DI: \url{https://packages.zendframework.com/docs/latest/manual/en/modules/zend.di.introduction.html}
    \end{itemize}
\end{frame}

\begin{frame}
	\begin{center}
        {\Huge Ačiū}
        \vskip1cm
        Atsiliepimai: \url{https://joind.in/8105}
	\end{center}
\end{frame}

\end{document}
