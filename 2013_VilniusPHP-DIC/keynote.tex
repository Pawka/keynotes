\documentclass[12pt,a4paper]{beamer}
\usepackage{graphicx}
\usepackage{hyperref}
\usepackage{lmodern}
\usepackage{listings}
\usepackage[utf8x]{inputenc}
\usepackage[L7x]{fontenc}
\usepackage[lithuanian]{babel}
\usepackage{minted}

% Theme
\usetheme{Antibes} 
\usecolortheme[RGB={155,192,12}]{structure} 

% Code highlight setup
\usemintedstyle{monokai}
\definecolor{codebg}{rgb}{0.10, 0.11, 0.08}
\newminted{php}{
    linenos,
    bgcolor=codebg,
    firstline=2,
    fontsize=\scriptsize,
    gobble=4
}
\definecolor{bashbg}{rgb}{0.95,0.95,0.95}
\newminted{bash}{
    bgcolor=bashbg,
    fontsize=\scriptsize,
    gobble=4
}

\author{Povilas Balzaravičius}
\title{Dependency Injection Containers}
\subtitle{VilniusPHP Susitikimas \#3}

\begin{document}
\section{Įžanga}
\begin{frame}
	\titlepage
\end{frame}

\begin{frame}{Kas aš toks?}
    \begin{itemize}
        \item Povilas Balzaravičius
        \item \href{https://twitter.com/pawka}{@Pawka}
        \item \href{https://github.com/pawka}{github.com/pawka}
        \item \href{https://linkedin.com/in/pawka}{linkedin.com/in/pawka}
        \item \href{http://pawka.linija.net}{pawka.linija.net}
    \end{itemize}
    \begin{center}
        \includegraphics[scale=0.4]{img/estina.png}
        \hskip1.5cm
        \includegraphics[scale=0.4]{img/zce.png}
        \hskip1.5cm
        \includegraphics[scale=0.75]{img/ktu.png}
    \end{center}
\end{frame}

\section{Dependency Injection}
\subsection{Kas yra Dependency Injection?}
\begin{frame}
	\begin{center}
        {\Huge Kas yra Dependency Injection?}
	\end{center}
\end{frame}


\begin{frame}[fragile]{Kuo blogas šis kodas?}
\begin{phpcode}
    <?php
    class FeedGenerator {

        protected $feed;

        public function __construct() {
            // Create feed every 6 hours.
            $this->feed = new SomeFeed('0 */6 * * *');
        }
    }
    //..
    $generator = new FeedGenerator;
\end{phpcode}
\pause
\begin{enumerate}
    \item Sudėtinga parašyti testą.
    \item Gali tekti keisti kodą modifikuojant Feed klasę.
    \item Teks modifikuoti FeedGenerator klasę, norint pakeisti SomeFeed vykdymo intervalą.
\end{enumerate}
\end{frame}

\begin{frame}[fragile]{Ar dabar geriau?}
\begin{phpcode}
    <?php
    class FeedGenerator {

        protected $feed;

        public function __construct($interval) {
            $this->feed = new SomeFeed($interval);
        }
    }
    //..
    $generator = new FeedGenerator('0 */6 * * *');
\end{phpcode}
\pause
\begin{enumerate}
    \item Sudėtinga parašyti testą.
    \item Gali tekti keisti kodą modifikuojant Feed klasę.
    \item SomeFeed klasė vis dar priklausoma nuo FeedGenerator.
\end{enumerate}
\end{frame}


\begin{frame}[fragile]{Ką daryti?}
    \begin{center}
        {\Huge Iškelti vidinių objektų kūrimą už klasės ribų!}
    \end{center}
\end{frame}


\begin{frame}[fragile]{Kodėl šis kodas geresnis?}
\begin{phpcode}
    <?php
    namespace Feed;

    class FeedGenerator {

        protected $feed;

        public function __construct(FeedInterface $feed) {
            $this->feed = $feed;
        }
    }
    //..
    $feed = new SomeFeed('0 */6 * * *');
    $generator = new FeedGenerator($feed);
\end{phpcode}
\begin{enumerate}
    \item Patogu rašyti testus (vietoj Feed galima paduoti mock objektą).
    \item \$feed objektas nepriklauso nuo FeedGenerator klasės.
    \item Modifikuojant FeedInterface klases, nereikės keisti FeedGenerator kodo.
\end{enumerate}
\end{frame}

\begin{frame}[fragile]{Sveiki, aš DI}
    \begin{center}
        {\Huge Štai Jums Dependency Injection!}
    \end{center}
\end{frame}

\begin{frame}[fragile]{Objektų kūrimas}
    DI naudojimui konteineris nereikalingas.
    \vskip0.5cm
\begin{phpcode}
    <?php
    $config = new \Doctrine\DBAL\Configuration();
    $connectionParams = array(
        'dbname' => 'mydb',
        'user' => 'user',
        'password' => 'secret',
        'host' => 'localhost',
        'driver' => 'pdo_mysql',
    );
    $conn = \Doctrine\DBAL\DriverManager::getConnection(
        $connectionParams, $config);
    $productService = new Product($conn);
    $interval = "0 */6 * * *";
    $feed = new \Feed\SomeFeed($interval);
    $feed->setProductService($productService);
    $feedGenerator = new \Feed\FeedGenerator($feed);
    //Maybe add objects to some registry.
    //..

\end{phpcode}
\end{frame}

\begin{frame}{Kas yra DI?}

    \begin{itemize}
        \item Projektavimo šablonas \emph{(design pattern)}.
        \item Aktualus tik objektiniame programavime.
        \item Atskiria komponentus \emph{(decoupling)}.
        \item Standartizuoja ir suteikia galimybę centralizuoti objektų kūrimą Jūsų sistemoje.
    \end{itemize}
\end{frame}

\begin{frame}[fragile]{DI būdai}
\begin{phpcode}
    <?php
    $feed = new Feed("0 */6 * * *");

    //Konstruktorius
    $generator = new FeedGenerator($feed);

    //Metodas (setter)
    $generator = new FeedGenerator;
    $generator->setFeed($feed);

    //Atributas
    $generator = new FeedGenerator;
    $generator->feed = $feed;
\end{phpcode}
\end{frame}

\section{DIC bibliotekos}
\begin{frame}
	\begin{center}
        {\footnotesize Kam kurti savo jei yra..}\\*
        \vskip0.5cm
        {\Huge DIC bibliotekos}
	\end{center}
\end{frame}

\begin{frame}{Reikalavimai DI konteineriui}
    \begin{itemize}
        \item Turi veikti greitai.
        \item Privalo dirbti su bet kokiu PHP objektu.
        \item Objektai neturi žinoti apie DI konteinerį.
        \item Nekurti objekto jei jis jau sukurtas.\footnote{Kai to reikia.}
    \end{itemize}
\end{frame}

\subsection{Pimple}
\begin{frame}
	\begin{center}
        {\Huge Pimple}
	\end{center}
\end{frame}

\begin{frame}{Pimple}
    A simple Dependency Injection Container for PHP 5.3.
    \begin{itemize}
        \item Autorius: Fabien Potencier
        \item \url{pimple.sensiolabs.org}
        \item Paprastas.
        \item Greitas.
        \item Integravimas - kelios eilutės.
    \end{itemize}
\end{frame}

\begin{frame}[fragile]{Darbas su objektais}
\begin{phpcode}
    <?php
    $container = new Pimple();
    // ..

    $container['feed.somefeed.interval'] = '0 */6 * * *';
    $container['feed.somefeed.class'] = '\Feed\SomeFeed';

    $container['feed.somefeed'] = function($c) {
        return new $c['feed.somefeed.class'](
            $c['feed.somefeed.interval']
        );
    };

    //Lazy loading
    $container['feed.generator.class'] = '\Feed\FeedGenerator';
    $container['feed.generator'] = $container->share(function($c) {
        return new $c['feed.generator.class']($c['feed.somefeed']);
    });

    // ..
    $manager = $container['feed.generator'];
\end{phpcode}
\end{frame}

\begin{frame}[fragile]{Išskaidymas į modulius}
\begin{phpcode}
    <?php
    class FeedContainer extends Pimple {

        public function __construct() {
            $this['feed.somefeed.interval'] = '0 */6 * * *';
            $this['feed.somefeed.class'] = '\Feed\SomeFeed';

            $this['feed.somefeed'] = function($c) {
                return new $c['feed.somefeed.class'](
                    $c['feed.somefeed.interval']
                );
            };

            $this['feed.generator.class'] = '\Feed\FeedGenerator';
            $this['feed.generator'] = $this->share(function($c) {
                return new $c['feed.generator.class']($c['feed.somefeed']);
            });
        }
    }
\end{phpcode}
\end{frame}

\begin{frame}[fragile]{Išskaidymas į modulius}
\begin{phpcode}
    <?php
    class GlobalContainer extends Pimple {

        public function __construct() {
            $this['feeds'] = $this->share(function($c) {
                return new FeedContainer;
            };
        }
    }

    //..
    $container = new GlobalContainer;
    //Change value
    $container['feeds']['feed.somefeed.interval'] = '0 */12 * * *';
    $generator = $container['feeds']['feed.generator'];
\end{phpcode}
\end{frame}


\begin{frame}[fragile]{Sparta}
    Konteinerio kūrimas
    \vskip0.5cm
\begin{bashcode}
    ./run-tests.sh 
    Pimple
    265K
    579K
    0.00086s
    DependencyInjection
    267K
    931K
    0.006124s
    DependencyInjection (dumped)
    688K
    738K
    0.000495s
\end{bashcode}
    \vskip0.25cm
    \scriptsize{Šaltinis: \url{https://gist.github.com/igorw/3833123}}
\end{frame}


\subsection{Symfony Dependency Injection Component}
\begin{frame}
	\begin{center}
        {\Huge Symfony Dependency Injection Component}
	\end{center}
\end{frame}

\begin{frame}{Symfony DIC}
    \begin{itemize}
        \item Symfony komponentas.
        \item \url{http://symfony.com/components}
        \item Daug galimybių.
        \item Greitas (teisingai naudojant).
        \item Pagal nutylėjimą naudojamas Symfony2 karkase.
    \end{itemize}
\end{frame}

\subsection{Zend Framework 2 Dependncy Injection}
\begin{frame}
	\begin{center}
        {\Huge Zend Framework 2 Dependncy Injection}
	\end{center}
\end{frame}

\section{Pabaiga}

\subsection{Išvados}
\begin{frame}{Ar verta naudoti DIC mažuose projektuose?}
    \begin{itemize}
        \item TAIP - nes DI verčia rašyti kodą teisingai.
    \end{itemize}
\end{frame}
\begin{frame}
	\begin{center}
        {\Huge Ačiū}
	\end{center}
\end{frame}

\end{document}
