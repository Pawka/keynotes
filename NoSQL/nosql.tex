\documentclass[12pt,a4paper]{beamer}
\usepackage[utf8x]{inputenc}
\usepackage{amsmath}
\usepackage{amsfonts}
\usepackage{amssymb}
\usepackage[L7x]{fontenc}
\usepackage[lithuanian]{babel}

\usetheme{Copenhagen} 
\usecolortheme[RGB={155,192,12}]{structure} 
%\useoutertheme{infolines} %adds infoline at the bottom

\author{Povilas Balzaravičius}
\title{NoSQL duomenų bazės}
\institute{M2 technologijos}

\begin{document}
\begin{frame}
	\titlepage
\end{frame}

%\begin{frame}
%	\frametitle{OUTLINE}
%		\tableofcontents[part=1,pausesections]
%\end{frame}

\begin{frame}{SQL}
	\textbf{SQL} - užklausų kalba, skirta manipuliavimui duomenimis reliacinėse duomenų bazėse.
\end{frame}

\begin{frame}{ACID}
	\textbf{ACID} - savybės, užtikrinančios, kad DBVS veikia patikimai.
	\vskip10pt
	\begin{itemize}
	\item Atomicity - užklausų transakcijos.
	\item Consistency \textit{(nuoseklumas)} - duomenys nebus sugadinti po transakcijos nutraukimo.
	\item Isolation - kitos operacijos negali prieiti prie duomenų, naudojamų transakcijos.
	\item Durability - užtikrinimas, kad įvykdytų užklausų duomenys nebus prarasti.
	\end{itemize}
\end{frame}

\begin{frame}{CAP}
	\textbf{CAP} - paskirstytų sistemų veikimo principai.
	\vskip10pt
	\begin{itemize}
		\item Consistency - užtikrina, kad įrašyti duomenys iškarto bus pasiekiami.
		\item Availability - informacijos pasiekiamumas (replikacijos tarp serverių).
		\item Partition tolerance - Duomenys gali būti nuskaitomi, jei dalies jų negalima gauti.
		\vskip20pt
		\item Vėlinimas - duomenų pateikimo sparta.
		\item Lankstumas \textit{(elasticity)} - downtime laikas.
	\end{itemize}
\end{frame}

%\subsection{NoSQL}
%\subsubsection{Kas tai yra?}
\begin{frame}{NoSQL - kas tai yra?}
	\textbf{NoSQL} - duomenų bazių tipas ir kartu judėjimas, aktyvus pastaruosius 2 metus.
	\vskip10pt
	\begin{itemize}
		\item Ne realiacinės duomenų bazės
		\item Iš anksto neapibrėžtos duomenų struktūros (lentelės)
		\item Nenaudojamos SQL užklausos
		\item Dažniausiai nepalaiko JOIN operacijų
	\end{itemize}
\end{frame}

\begin{frame}{Panaudojimas}
	Pagrindinė paskirtis - sistemos apkrovimo sumažinimas.
	\vskip10pt
	\begin{itemize}
		\item Duomenų saugojimas paskirstytose sistemose \textit{(horisontal scaling)}
		\item Didelė aparta
		\item Naudojama ne tik web aplikacijose
	\end{itemize}
\end{frame}


\begin{frame}{Tipai}
	\begin{itemize}
	\item<1-> Document-oriented - \textit{saugomas duomenų rinkinys, leidžia duomenis atrinkti pagal laukus.}
	\item<2-> Key-value - \textit{identifikuojama pagal vieną raktą.}
	\item<3-> Column-oriented
	\item<4-> ...
	\item<4-> Object-oriented
	\item<4-> XML
	\item<4-> Graph
	\item<4-> ...
	\end{itemize}
\end{frame}


\begin{frame}{Populiariausios duomenų bazės}
	\begin{itemize}
	\item Memcachedb
	\item Redis
	\item MongoDB
	\item CouchDB
	\item Cassandra
	\item Voldemort
	\item ...
	\end{itemize}
\end{frame}


\begin{frame}{MemcacheDB}
	\begin{itemize}
	\item Key-value
	\item Orientuotas į spartą
	\item Naudojam mes :-) 
	\item Duomenys saugomi RAM
	\end{itemize}
\end{frame}


\begin{frame}{Redis}
	\begin{itemize}
	\item Key-value
	\item Orientuotas į spartą
	\item Memcached atitikmuo
	\item Leidžia išsaugoti būseną į diską
	\item Master-slave replikacijos (gerai skaitymui, blogai rašymui)
	\end{itemize}
\end{frame}


\begin{frame}{MongoDB}
	\begin{itemize}
	\item Document-oriented
	\item Orientuojasi į spartą ir funkcionalumą
	\item MongoDB ODM (Doctrine) for PHP
	\item Palaiko ryšius tarp duomenų
	\item Duomenys saugomi JSON
	\item Master-slave replikacijos
	\item Sourceforge (?)
	\end{itemize}
\end{frame}


\begin{frame}{CouchDB}
	\begin{itemize}
	\item Document-oriented
	\item Turi viewsus
	\item Duomenys saugomi JSON
	\item REST API
	\item Master-master replikacijos. Atskiros kopijos sinchronizuojasi tarpusavyje.
	\end{itemize}
\end{frame}


\begin{frame}{Cassandra}
	\begin{itemize}
	\item Column-oriented
	\item Didelė sparta
	\item Sukūrė Facebook
	\item Naudoja Digg, Twitter, Reddit (iš memcached)
	\end{itemize}
\end{frame}


\begin{frame}{Voldemort}
	\begin{itemize}
	\item Key-value
	\item Automatinės replikacijos
	\item Didelė sparta
	\item Naudoja ir sukūrė LinkedIn
	\end{itemize}
\end{frame}



\begin{frame}{Trūkumai ir problemos}
	\begin{itemize}
	\item Naujos sistemos - gali būti nepatikimos
	\item NoSQL - nereiškia, kad nėra SQL injection
	\item Ne visi įrankiai yra gerai išvystyti \textit{(pvz. pymongo)}
	\end{itemize}
\end{frame}



\begin{frame}{Apibendrinimas}
	\begin{itemize}
	\item Nereikia stengtis pakeisti MySQL (neįsivaizduoju kaip tai įmanoma)
	\item Naudoti, kai reikia paskirstyt duomenis per kelis serverius. Bet mums artimiausiu metu tai negresia.
	\item Galima kurti spartos reikalaujančius modulius, naudojančius konkrečią duomenų bazę.
	\end{itemize}
\end{frame}

\begin{frame}
	\begin{center}
	Ačiū
	\end{center}
\end{frame}


\end{document}