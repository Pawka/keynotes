\documentclass[12pt,a4paper]{beamer}
% \usepackage{amsmath}
% \usepackage{amsfonts}
% \usepackage{amssymb}
\usepackage{graphicx}
\usepackage{hyperref}
\usepackage{lmodern}
\usepackage{listings}
\usepackage[utf8x]{inputenc}
\usepackage[L7x]{fontenc}
\usepackage[lithuanian]{babel}

\usetheme{Antibes} 
\usecolortheme[RGB={155,192,12}]{structure} 

\author{Povilas Balzaravičius}
\title{Seni projektai, nauji įrankiai}
\subtitle{VilniusPHP Susitikimas \#1}

\begin{document}
\begin{frame}
	\titlepage
\end{frame}

\section{Įžanga}
\begin{frame}{Kas aš toks?}
    \begin{itemize}
        \item Povilas Balzaravičius
        \item \href{https://twitter.com/pawka}{@Pawka}
        \item \href{https://github.com/pawka}{github.com/pawka}
        \item \href{https://linkedin.com/in/pawka}{linkedin.com/in/pawka}
        \item \href{http://pawka.linija.net}{pawka.linija.net}
    \end{itemize}
    \begin{center}
        \includegraphics[scale=0.4]{img/estina.png}
        \hskip1cm
        \includegraphics[scale=0.4]{img/zce-php5-3-logo.png}
    \end{center}
\end{frame}

\subsection{Seni projektai}
\begin{frame}[fragile]

    {\Huge Seni projektai}
\end{frame}

\begin{frame}{Kas jie tokie?}
    \begin{itemize}
        \item PHP 4.x/5.x.
        \item Kodas >= 4 metų senumo.
        \item Niekur nematytas kodo stilius(-ai).
        \item Nenaudojamas žmonijai žinomas karkasas.
        \item \textbf{include}, \textbf{require} ir draugai.
    \end{itemize}
\end{frame}

\begin{frame}{Kylančios problemos}
    \begin{itemize}
        \item Didžiulės sąnaudos tvarkingai perrašyti kodą.
        \item Naujo funkcionalumo pridėjimas reikalauja daug laiko.
        \item Kažką pajudinus viskas griūna.
        \item Šlykštoka dirbti\dots
    \end{itemize}
\end{frame}

\section{Projektų gaivinimas}
\begin{frame}

    {\Huge Pradedam!}\\
\end{frame}

\subsection{Kodo stilius}
\begin{frame}

    {\Huge Stilius}\\
    Naujas projektas - naujas programavimo stilius.
\end{frame}

\begin{frame}[fragile]
    \frametitle{Standartas}

    {\Huge PSR}\\
\end{frame}

\begin{frame}
    \frametitle{PSR Standartas}
    {\large PHP Specification Request - programavimo stiliaus rekomendacija.}
    \vskip15pt
    \pause
    Sudaro trys dokumentai:
    \begin{description}
        \item[PSR-0] Autoload standartas.
        \pause
        \item[PSR-1] ``Basic Coding Standard'' - Koduotė, PHP \textit{tagai}, konstantos, klasių ir metodų pavadinimai, \dots
        \pause
        \item[PSR-2] ``Coding Style Guide'' - Tarpai, skliausteliai ir kableliai :-)
        \pause
    \end{description}
    \vskip10pt
    {\small Standarto aprašymas: \href{https://github.com/php-fig/fig-standards/}{github.com/php-fig/fig-standards/}}
\end{frame}

\begin{frame}[fragile]
    \frametitle{Sprendimas}

    {\Huge php-cs-fixer}
\end{frame}

\begin{frame}
    \frametitle{PHP Coding Standards Fixer}
    Įrankis skirtas kodo stiliaus tvarkymui pagal \textbf{PSR-1} ir \textbf{PSR-2} standartus.
    \begin{itemize}
        \item Autorius: Fabien Potencier
        \item \url{http://cs.sensiolabs.org/}
        \item Galimybė tvarkyti tik tam tikras sritis (identacija, skliaustų išdėstymas, \dots)
        \item \dots ir/arba naudoti paruoštas konfigūracijas (sf20, sf21, magento, default).
    \end{itemize}
\end{frame}

\begin{frame}
    \frametitle{PHP Coding Standards Fixer - naudojimas}
    {\small
        \begin{enumerate}
            \item {\color{orange}sudo wget http://cs.sensiolabs.org/get/php-cs-fixer.phar -O /usr/local/bin/php-cs-fixer}
            \item {\color{orange}sudo chmod a+x /usr/local/bin/php-cs-fixer}
            \item {\color{orange}php php-cs-fixer.phar fix /path/to/dir}
            \item Džiaugiamės tvarkingu kodu :-) \pause(dažniausiai)
        \end{enumerate}
    }
\end{frame}
% TODO - add console sceenshot.

\subsection{Išorinės bibliotekos}
\begin{frame}[fragile]

    {\Huge Išorinės bibliotekos}\\
    Naudojam tai, kas jau sukurta.
\end{frame}

\begin{frame}
    \frametitle{Problemos}
    \begin{itemize}
        \item Nepatogu įtrauktį į kodą.
        \item Gali priklausyti nuo kitų bibliotekų.
        \item Saugomos kartu su projekto kodu.
        \item Nepatogu atnaujinti.
    \end{itemize}
\end{frame}

\begin{frame}[fragile]
    \frametitle{Sprendimas}

    {\Huge Composer}
\end{frame}

\begin{frame}
    \frametitle{Composer}

    \begin{itemize}
        \item Autorius: Jordi Boggiano
        \item \url{http://getcomposer.org/}
        \item Suranda, instaliuoja, atnaujina ir \textit{autoloadina} paketus.
    \end{itemize}
\end{frame}

\begin{frame}
    \frametitle{Composer - naudojimas}
        \begin{enumerate}
            \item {\color{orange}curl -s https://getcomposer.org/installer | php} 
            \item Parsiunčiamas \textbf{composer.phar}
            \item {\color{orange}php composer.phar <komanda>}
        \end{enumerate}
\end{frame}

\begin{frame}
    \frametitle{Composer - paketų instaliavimas}

    \begin{enumerate}
        \item Susirandam paketą ir jo versiją:\\{\color{orange}php composer.phar search kažkas} arba \url{packagist.org}.
        \item Pvz. \textbf{doctrine/dbal 2.3.0}.
        \item Įtraukiam paketą į \textbf{composer.json}.
        \item {\color{orange}php composer.phar install} (arba {\color{orange}update}).
        \item Parsiųs paketus ir sugeneruos Autoload failus.
    \end{enumerate}
\end{frame}

\begin{frame}

    \includegraphics[scale=0.3]{img/composer.png}
\end{frame}

\begin{frame}
    \frametitle{Composer - naudojimas}

    composer.json pavyzdys
    \begin{center}
        \includegraphics[scale=0.5]{img/composer_json.png}
    \end{center}
    \pause
    O tada\dots\\
    \begin{center}
        \includegraphics[scale=0.5]{img/autoload.png}
    \end{center}
\end{frame}

\begin{frame}[fragile]

    {\Huge Iš ko rinktis?}
\end{frame}

\begin{frame}
    \frametitle{Symfony Components}
    \url{symfony.com/components}
    {\scriptsize
        \begin{columns}[t]
            \begin{column}{5cm}
                \begin{itemize}
                    \item BrowserKit
                    \item ClassLoader
                    \item Config
                    \item Console
                    \item CssSelector
                    \item DependencyInjection
                    \item DomCrawler
                    \item EventDispatcher
                    \item Finder
                    \item Form
                    \item HttpFoundation
                \end{itemize}
            \end{column}
            \begin{column}{5cm}
                \begin{itemize}
                    \item HttpKernel	
                    \item Locale	
                    \item Process	
                    \item Routing	
                    \item Security	 	
                    \item Serializer	 	
                    \item Templating	
                    \item Translation	 	
                    \item Validator	 	
                    \item Yaml
                \end{itemize}
            \end{column}
        \end{columns}
    }
\end{frame}
\begin{frame}
    \frametitle{Zend Framework 2}

    {\Huge 48 atskiri komponentai}\\
    \url{framework.zend.com}
\end{frame}

\section{Mano pasirinkimas}
\begin{frame}[fragile]

    {\Huge Kaip dirbu aš?}
\end{frame}

\begin{frame}
    \frametitle{Darbo planas}
    \begin{enumerate}
        \item Stiliaus sutvarkymas.
        \item Composer paruošimas.
        \item DBAL (vienareikšmiškai \textbf{doctrine/dbal}).
        \item Dependency Injection Container.
        \item \textit{Routing'o} įdiegimas.
        \item Naujo kodo rašymas teisinga tvarka.
    \end{enumerate}
\end{frame}

\subsection{Dependency Injection Container}
\begin{frame}[fragile]

    {\Huge Dependency Injection Container}
\end{frame}

\begin{frame}
    \frametitle{Kam reikalingas?}

    \begin{itemize}
        \item Objektų kūrimo taisyklės saugomos vienoje vietoje.
        \item \textit{Lazy Loading} - objektai kuriami tik tada, kai jų prireikia.
    \end{itemize}

    \begin{block}{Fabien Potencier}
        \dots\ when you need to manage a lot of different objects with a lot of dependencies, a Dependency Injection Container can be really helpful (think of a framework for instance).
    \end{block}

    Pasiskaitymui: \href{http://fabien.potencier.org/article/11/what-is-dependency-injection}{``What is Dependency Injection''} by @fabpot.
    
\end{frame}
\begin{frame}
    \frametitle{Pimple}
    A simple Dependency Injection Container for PHP 5.3
    \begin{itemize}
        \item Autorius: Fabien Potencier
        \item \url{pimple.sensiolabs.org}
        \item Paprastas
        \item Greitas
        \item Integravimas - kelios eilutės.
        \item Primityvus (pvz. lyginant su Symfony DependencyInjection)
    \end{itemize}
\end{frame}
\subsection{Routing}
\begin{frame}[fragile]

    {\Huge Routing}
\end{frame}
\begin{frame}
    \frametitle{Kam reikalingas?}

    \begin{itemize}
        \item Iškviesti reikiamą \textit{Controller'io Actions'ą} pagal URL.
        \item Iškviečiant priskirti kintamuosius iš URL.
        \item URL generatorius pagal \textit{route'o} pavadinimą.
        \item Dažnai senesni projektai neturi :-(
    \end{itemize}
\end{frame}
\begin{frame}
    \frametitle{Symfony Routing Component}

    \begin{itemize}
        \item \url{symfony.com/doc/current/components/routing/}
        \item Integravimas - kelios eilutės.
        \item Sprendžia visas anksčiau aprašytas problemas.
    \end{itemize}
\end{frame}
\subsection{Teisingas kodas}
\begin{frame}[fragile]

    {\Huge Teisingas kodas}
\end{frame}

\begin{frame}
    \frametitle{Service-oriented architecture}

    \begin{block}{Symfony Glossary}
        A Service is a generic term for any PHP object that performs a specific
        task. A service is usually used "globally", such as a database
        connection object or an object that delivers email messages. In
        Symfony2, services are often configured and retrieved from the service
        container. An application that has many decoupled services is said to
        follow a service-oriented architecture.
    \end{block}
\end{frame}
\begin{frame}
    \frametitle{SOA Robotikos taisyklės}
    \begin{itemize}
        \item Modeliai - žemiausias lygmuo, vieni apie kitus nežino.
        \item Servisas žino apie jam priklausančius modelius.
        \item Servisas gali žinoti apie kitus servisus.
        \item Kontroleriai nežino nieko apie modelius ir naudoja TIK servisus.
    \end{itemize}
\end{frame}

\section{Pabaiga}
\begin{frame}
    \frametitle{Apibendrinimas}

    \begin{itemize}
        \item Naują kodą rašykite tvarkingai.
        \item Seną kodą perrašinėkite nedidelėmis dalimis ir tik tada, kai to reikia.
        \item Naudokite jau sukurtus įrankius.
        \item Sugaišit laiko sutvarkymui, bet laimėsit kurdami naujas funkcijas.
    \end{itemize}
\end{frame}
\begin{frame}
	\begin{center}
        {\Huge Ačiū}
	\end{center}
\end{frame}

\end{document}
